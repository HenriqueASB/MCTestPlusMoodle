\documentclass[10pt,brazil,a4paper]{exam}

\usepackage[utf8]{inputenc}
%\usepackage[portuguese]{babel}
\usepackage[brazilian]{babel}
\usepackage[table,xcdraw]{xcolor}

\usepackage[hidelinks]{hyperref}
%cachedir=./tmp/
%outputdir=./tmp/
%\usepackage[outputdir=tmp]{minted}
%\newcommand{\mintedoptions}{cachedir=./,outputdir=./tmp/_minted/}
%\usepackage[\mintedoptions]{minted}
\usepackage[cache=false]{minted}

\usepackage{algorithmicx}
\usepackage[noend]{algpseudocode}

%sudo apt install linuxbrew-wrapper
%sudo apt install texlive-extra-utils
%sudo apt install texlive-font-utils

\usepackage[pdftex]{graphicx}
\usepackage[dvips]{graphicx}
\usepackage{epstopdf}
\graphicspath{{./tmp/}}
\DeclareGraphicsExtensions{.pdf,.jpeg,.png,.eps}
\usepackage[outdir=./]{epstopdf}

\usepackage{listings} % para códigos

\usepackage[top=5mm, bottom=0mm, left=10mm, right=10mm]{geometry}
%\textheight=6in
\usepackage{tikz}
\usepackage{enumitem}
%\usepackage[shortlabels]{enumerate}
\usepackage{multirow}
\usepackage{amsmath}
\usepackage{amssymb,amsthm}
\usepackage{changepage,ifthen}
\usepackage{verbatim}
\usepackage{tabularx}
\usepackage{amsfonts}
\usepackage{multicol}

\usepackage{mathtools}
\usepackage{siunitx}
\makeatletter
\def\zz\ignorespaces{\@ifnextchar-{}{\phantom{-}}}
\newcolumntype{C}{>{\zz}{c}}
\makeatother

%\usepackage[T1]{fontenc}
%\usepackage{times}         
%\usepackage{multido}  % border
%\usepackage{pst-barcode}

%\setlength{\textwidth}{185mm}
%\setlength{\oddsidemargin}{-0.5in}
%\setlength{\evensidemargin}{0in}
%\setlength{\columnsep}{8mm}
%\setlength{\topmargin}{-28mm}
%\setlength{\textheight}{287mm}
%\setlength{\itemsep}{0in}
\setlength{\paperheight}{881pt}

\newcommand*\varhrulefill[1][0.4pt]{\leavevmode\leaders\hrule height#1\hfill\kern0pt}
\def\drawLines#1{{\color{lightgray}\foreach \x in {1,...,#1}{\par\vspace{2mm}\noindent\hrulefill}}}       
% cyan
\newcommand{\qid}[1]{\textcolor{white}{{\tiny #1}\hspace{0mm}}} %question identification (invisible text #1)
\usepackage[table,xcdraw]{xcolor}

%\renewcommand{\thefootnote}{\color{white}{\faGears}}
\renewcommand{\thefootnote}{\fnsymbol{footnote}}

\definecolor{bubbles}{rgb}{0.71, 1.0, 1.0}

\begin{document}
\pagestyle{empty}

\pagenumbering{gobble}
%\newgeometry{bottom=0.1cm}


%%%%%%%%%%%% algorithms by py2tex https://github.com/zhafen/py2tex

%\NeedsTeXFormat{LaTeX2e}
%\ProvidesPackage{py2tex}[09/05/2019 v0.0 Macros for py2tex]

                           %~~~~~New commands~~~~~%

%For bold sans serif
\newcommand \textbs[1]{\textsf{\textbf{#1}}}

\algrenewcommand\algorithmicprocedure{{\textbs{Função}}}
\algrenewcommand\algorithmicfunction{{\textbs{function}}}
\algrenewcommand\algorithmicif{{\textbs{Se}}}
\algrenewcommand\algorithmicthen{{\textbs{Então}}}
\algrenewcommand\algorithmicelse{{\textbs{Senão}}}
\algrenewcommand\algorithmicwhile{{\textbs{Enquanto}}}
\algrenewcommand\algorithmicdo{{\textbs{Faça}}}
\algrenewcommand\algorithmicend{{\textbs{Fim}}}
\algrenewcommand\algorithmicrepeat{{\textbs{Faça}}}
\algrenewcommand\algorithmicuntil{{\textbs{Enquanto}}}
\algrenewcommand\algorithmicreturn{\State{{\textbs{Retorna}}}}

\algblock[Name]{Start}{End}
\algblockdefx[NAME]{START}{END}%
[2][Unknown]{Start #1(#2)}%
{Ending}
\algblockdefx[NAME]{}{OTHEREND}%
[1]{Until (#1)}

                           %~~~~~Operators~~~~~%
\newcommand{\PyAdd}[0]{+}
\newcommand{\PyAnd}[0]{\textbf{ and }}
\newcommand{\PyAnnotation}[2]{\ensuremath{\textbs{variable }#1\colon \texttt{#2}}}
\newcommand{\PyArg}[1]{\texttt{#1}}
\newcommand{\PyArgAnnotation}[2]{\ensuremath{\texttt{#1} \colon \texttt{#2}}}
\newcommand{\PyArgSep}[0]{,~}
\newcommand{\PyAssign}[2]{\ensuremath{#1 \gets #2}}
\newcommand{\PyAssignSep}[0]{,~}

\newcommand{\PyCall}[2]{
	\ifnum\pdfstrcmp{#1}{print}=0
  	\textbs{output~} { \tt#2}
	\else
		\textsc{#1}( {\tt#2} )
	\fi
}
\newcommand{\PyCallSep}[0]{,~}

\newcommand{\PyDiv}[0]{\mathop{\texttt{/}}}

\newcommand{\PyEq}[0]{=}
\newcommand{\PyExpr}[1]{\ensuremath{#1}}

\newcommand{\PyFalse}[0]{\texttt{false}}

\newcommand{\PyGt}[0]{>}
\newcommand{\PyGtE}[0]{\ge}

%\newcommand{\PyIn}[0]{}
%\newcommand{\PyIs}[0]{}
%\newcommand{\PyIsNot}[0]{}

\newcommand{\PyList}[1]{[#1]}
\newcommand{\PyListSep}[0]{, }
\newcommand{\PyLt}[0]{<}
\newcommand{\PyLtE}[0]{\le}

\newcommand{\PyMod}[0]{\mathop{\textsc{ mod }}}
\newcommand{\PyMult}[0]{\times}

\newcommand{\PyOr}[0]{\textbf{ or }}

\newcommand{\PyName}[1]{\texttt{#1}}
%\newcommand{\PyNone}[0]{\textt{}}
\newcommand{\PyNot}[1]{\textbf{ not }#1}
\newcommand{\PyNotEq}[0]{\neq}
%\newcommand{\PyNotIn}[0]{}
\newcommand{\PyNum}[1]{\texttt{#1}}

\newcommand{\PyPar}[1]{\left ( #1 \right )}
\newcommand{\PyPow}[0]{\texttt{^}}

\newcommand{\PyStr}[1]{\texttt{``#1''}}
\newcommand{\PySub}[0]{-}
\newcommand{\PySubscript}[2]{#1[#2]}

\newcommand{\PyTrue}[0]{\texttt{true}}

\newcommand{\PyUAdd}[1]{+ #1}
\newcommand{\PyUSub}[1]{- #1}


% Linhas verticais

%\PassOptionsToPackage{noend}{algpseudocode}% comment out if want end's to show
%\usepackage{algpseudocode}% http://ctan.org/pkg/algorithmicx

\errorcontextlines\maxdimen

% begin vertical rule patch for algorithmicx (http://tex.stackexchange.com/questions/144840/vertical-loop-block-lines-in-algorithmicx-with-noend-option)
\makeatletter
% start with some helper code
% This is the vertical rule that is inserted
\newcommand*{\algrule}[1][\algorithmicindent]{\makebox[#1][l]{\hspace*{.5em}\vrule height .75\baselineskip depth .25\baselineskip}}%

\newcount\ALG@printindent@tempcnta
\def\ALG@printindent{%
    \ifnum \theALG@nested>0% is there anything to print
        \ifx\ALG@text\ALG@x@notext% is this an end group without any text?
            % do nothing
            \addvspace{-3pt}% FUDGE for cases where no text is shown, to make the rules line up
        \else
            \unskip
            % draw a rule for each indent level
            \ALG@printindent@tempcnta=1
            \loop
                \algrule[\csname ALG@ind@\the\ALG@printindent@tempcnta\endcsname]%
                \advance \ALG@printindent@tempcnta 1
            \ifnum \ALG@printindent@tempcnta<\numexpr\theALG@nested+1\relax% can't do <=, so add one to RHS and use < instead
            \repeat
        \fi
    \fi
    }%

%\usepackage{etoolbox}
% the following line injects our new indent handling code in place of the default spacing
\patchcmd{\ALG@doentity}{\noindent\hskip\ALG@tlm}{\ALG@printindent}{}{\errmessage{failed to patch}}
\makeatother
% end vertical rule patch for algorithmicx

\noindent\Huge{MCTest}\normalsize\vspace{5mm}\\
\noindent\textbf{Topic:} Topico Exemplo\\
\noindent\rule{\textwidth}{0.8pt}\\
\noindent\textbf{Count:} 1\\
\noindent\textbf{Short Description:} QE 4\\
\noindent\textbf{Group:} \\
\noindent\textbf{Type:} QM\\
\noindent\textbf{Difficulty:} 1\\
\noindent\textbf{Bloom taxonomy:} remember\\
\noindent\textbf{Last update:} 2023-12-07\\
\noindent\textbf{Who created:} fzampirolli@ufabc.edu.br\\
\noindent\textbf{URL:} \url{http://127.0.0.1:8000/topic/question/2446/update/}\\
\noindent\textbf{Parametric:} NO\\

\hspace{-15mm}{\small {\color{green}\#2446}} \hspace{-1mm} 1. Quarta questão.
\n\n

\vspace{2mm}\begin{oneparchoices}\hspace{-3mm}
\end{oneparchoices}\vspace{0mm}\\
\noindent\rule{\textwidth}{0.8pt}\\
\noindent\textbf{Count:} 2\\
\noindent\textbf{Short Description:} QE2\\
\noindent\textbf{Group:} \\
\noindent\textbf{Type:} QM\\
\noindent\textbf{Difficulty:} 1\\
\noindent\textbf{Bloom taxonomy:} remember\\
\noindent\textbf{Last update:} 2023-04-17\\
\noindent\textbf{Who created:} fzampirolli@ufabc.edu.br\\
\noindent\textbf{URL:} \url{http://127.0.0.1:8000/topic/question/2444/update/}\\
\noindent\textbf{Parametric:} NO\\

\hspace{-15mm}{\small {\color{green}\#2444}} \hspace{-1mm} 1. Segunda Questao
\n\n

\vspace{2mm}\begin{oneparchoices}\hspace{-3mm}
\choice \hspace{-2.0mm}{\tiny{\color{red}*2}}3a alternativa criada
\choice \hspace{-2.0mm}{\tiny{\color{red}*4}}5a alternativa criada
\choice \hspace{-2.0mm}{\tiny{\color{red}*3}}4a alternativa criada
\choice \hspace{-2.0mm}{\tiny{\color{red}*1}}2a alternativa criada
\choice \hspace{-2.0mm}{\tiny{\color{blue}\#0}}1a alternativa criada
\end{oneparchoices}\vspace{0mm}\\
\noindent\rule{\textwidth}{0.8pt}\\
\noindent\textbf{Count:} 3\\
\noindent\textbf{Short Description:} QE1\\
\noindent\textbf{Group:} \\
\noindent\textbf{Type:} QM\\
\noindent\textbf{Difficulty:} 1\\
\noindent\textbf{Bloom taxonomy:} remember\\
\noindent\textbf{Last update:} 2023-04-17\\
\noindent\textbf{Who created:} fzampirolli@ufabc.edu.br\\
\noindent\textbf{URL:} \url{http://127.0.0.1:8000/topic/question/2443/update/}\\
\noindent\textbf{Parametric:} NO\\

\hspace{-15mm}{\small {\color{green}\#2443}} \hspace{-1mm} 1. Primeira Questao - Sempre a primeira alternativa criada eh a correta.
\n\n

\vspace{2mm}\begin{oneparchoices}\hspace{-3mm}
\choice \hspace{-2.0mm}{\tiny{\color{blue}\#0}}1a alternativa criada
\choice \hspace{-2.0mm}{\tiny{\color{red}*3}}4a alternativa criada
\choice \hspace{-2.0mm}{\tiny{\color{red}*1}}2a alternativa criada
\choice \hspace{-2.0mm}{\tiny{\color{red}*4}}5a alternativa criada
\choice \hspace{-2.0mm}{\tiny{\color{red}*2}}3a alternativa criada
\end{oneparchoices}\vspace{0mm}\\
\noindent\rule{\textwidth}{0.8pt}\\
\noindent\textbf{Count:} 4\\
\noindent\textbf{Short Description:} QE3\\
\noindent\textbf{Group:} \\
\noindent\textbf{Type:} QM\\
\noindent\textbf{Difficulty:} 1\\
\noindent\textbf{Bloom taxonomy:} remember\\
\noindent\textbf{Last update:} 2023-04-17\\
\noindent\textbf{Who created:} fzampirolli@ufabc.edu.br\\
\noindent\textbf{URL:} \url{http://127.0.0.1:8000/topic/question/2445/update/}\\
\noindent\textbf{Parametric:} NO\\

\hspace{-15mm}{\small {\color{green}\#2445}} \hspace{-1mm} 1. Terceira questao
\n\n

\vspace{2mm}\begin{oneparchoices}\hspace{-3mm}
\choice \hspace{-2.0mm}{\tiny{\color{blue}\#0}}1a alternativa criada
\choice \hspace{-2.0mm}{\tiny{\color{red}*2}}3a alternativa criada
\choice \hspace{-2.0mm}{\tiny{\color{red}*4}}5a alternativa criada
\choice \hspace{-2.0mm}{\tiny{\color{red}*1}}2a alternativa criada
\choice \hspace{-2.0mm}{\tiny{\color{red}*3}}4a alternativa criada
\end{oneparchoices}\vspace{0mm}\\
\newpage\\
\end{document}